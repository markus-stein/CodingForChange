% Options for packages loaded elsewhere
\PassOptionsToPackage{unicode}{hyperref}
\PassOptionsToPackage{hyphens}{url}
%
\documentclass[
]{article}
\usepackage{lmodern}
\usepackage{amsmath}
\usepackage{ifxetex,ifluatex}
\ifnum 0\ifxetex 1\fi\ifluatex 1\fi=0 % if pdftex
  \usepackage[T1]{fontenc}
  \usepackage[utf8]{inputenc}
  \usepackage{textcomp} % provide euro and other symbols
  \usepackage{amssymb}
\else % if luatex or xetex
  \usepackage{unicode-math}
  \defaultfontfeatures{Scale=MatchLowercase}
  \defaultfontfeatures[\rmfamily]{Ligatures=TeX,Scale=1}
\fi
% Use upquote if available, for straight quotes in verbatim environments
\IfFileExists{upquote.sty}{\usepackage{upquote}}{}
\IfFileExists{microtype.sty}{% use microtype if available
  \usepackage[]{microtype}
  \UseMicrotypeSet[protrusion]{basicmath} % disable protrusion for tt fonts
}{}
\makeatletter
\@ifundefined{KOMAClassName}{% if non-KOMA class
  \IfFileExists{parskip.sty}{%
    \usepackage{parskip}
  }{% else
    \setlength{\parindent}{0pt}
    \setlength{\parskip}{6pt plus 2pt minus 1pt}}
}{% if KOMA class
  \KOMAoptions{parskip=half}}
\makeatother
\usepackage{xcolor}
\IfFileExists{xurl.sty}{\usepackage{xurl}}{} % add URL line breaks if available
\IfFileExists{bookmark.sty}{\usepackage{bookmark}}{\usepackage{hyperref}}
\hypersetup{
  pdftitle={Minecraft and Programming},
  pdfauthor={Markus Stein},
  hidelinks,
  pdfcreator={LaTeX via pandoc}}
\urlstyle{same} % disable monospaced font for URLs
\usepackage[margin=1in]{geometry}
\usepackage{graphicx}
\makeatletter
\def\maxwidth{\ifdim\Gin@nat@width>\linewidth\linewidth\else\Gin@nat@width\fi}
\def\maxheight{\ifdim\Gin@nat@height>\textheight\textheight\else\Gin@nat@height\fi}
\makeatother
% Scale images if necessary, so that they will not overflow the page
% margins by default, and it is still possible to overwrite the defaults
% using explicit options in \includegraphics[width, height, ...]{}
\setkeys{Gin}{width=\maxwidth,height=\maxheight,keepaspectratio}
% Set default figure placement to htbp
\makeatletter
\def\fps@figure{htbp}
\makeatother
\setlength{\emergencystretch}{3em} % prevent overfull lines
\providecommand{\tightlist}{%
  \setlength{\itemsep}{0pt}\setlength{\parskip}{0pt}}
\setcounter{secnumdepth}{-\maxdimen} % remove section numbering
\ifluatex
  \usepackage{selnolig}  % disable illegal ligatures
\fi

\title{Minecraft and Programming}
\author{Markus Stein}
\date{09 Febraury 2020}

\begin{document}
\maketitle

This are some notes about coding and playing Minecraft.

\hypertarget{is-it-possible-to-play-minecraft-with-r}{%
\section{\texorpdfstring{Is it possible to play \emph{minecraft} with
\texttt{R}?}{Is it possible to play minecraft with R?}}\label{is-it-possible-to-play-minecraft-with-r}}

\hypertarget{books}{%
\subsection{Books}\label{books}}

\begin{itemize}
\tightlist
\item
  \href{https://nostarch.com/programwithminecraft}{Learn to Program with
  Minecraft: Transform Your World with the Power of Python}
\end{itemize}

\hypertarget{r-packages}{%
\subsection{R packages}\label{r-packages}}

\begin{itemize}
\tightlist
\item
  \href{https://ropenscilabs.github.io/miner_book/index.html}{\texttt{miner}
  - R Programming with Minecraft}
\end{itemize}

Setting up your own server:\\
a. BuildTools - \url{https://www.spigotmc.org/wiki/buildtools/}\\
b. Spigot Installation -
\url{https://www.spigotmc.org/wiki/spigot-installation/}\\
c.~Raspberry Juice plugin -
\url{https://dev.bukkit.org/projects/raspberryjuice}

\hypertarget{blockdown-simulator}{%
\subsection{Blockdown Simulator}\label{blockdown-simulator}}

\hypertarget{covid-19-is-spreading-to-minecraft}{%
\subsubsection{Covid-19 is spreading to
Minecraft}\label{covid-19-is-spreading-to-minecraft}}

This is a map to simulate a highly infectious Zombie pandemic spreading
among villagers. You can observe, experience social distancing or run a
hospital during this pandemic.

\begin{itemize}
\item
  \url{https://www.blockdown.net/}
\item
  \url{https://www.minecraft.net/en-us/article/let-s-play--blockdown-simulator}
\end{itemize}

How to install maps?

\begin{itemize}
\tightlist
\item
  \url{https://minecraft.gamepedia.com/Tutorials/Map_downloads}
\end{itemize}

\end{document}
